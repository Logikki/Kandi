\documentclass[utf8,bachelor,finnish]{bachelor}
% If you are writing a Master's Thesis, use the following instead:
%\documentclass[utf8,english]{gradu3}
% overleaf pohja : https://www.overleaf.com/project/63e5ffa6c6c53ac541c21a89
\usepackage{graphicx} % for including pictures

\usepackage{amsmath} % useful for math (optional)
\usepackage[
  separate-uncertainty = true,
  multi-part-units = repeat
]{siunitx}
\sisetup{output-decimal-marker = {,}}
\usepackage{booktabs} % good for beautiful tables

% NOTE: This must be the last \usepackage in the whole document!
\usepackage[bookmarksopen,bookmarksnumbered,linktocpage]{hyperref}
\usepackage{float}
\addbibresource{references.bib} % The file name of your bibliography database

\begin{document}

\title{Miten äly- ja urheilukellot suoriutuvat niille sunnitelluista tehtävistä}
\translatedtitle{\LaTeX-tutkielmapohjan {gradu3} käyttö}
\studyline{Mathematical Information Technology}
\avainsanat{%
Puettava teknologia, Älykellot, Urheilukellot
}
\keywords{\LaTeX, {gradu3}, Master's Theses, Bachelor's Theses, user's guide}
\tiivistelma{%
 Urheilukellot ovat yleistyneet viime vuosina paljon. Laitteet tekevät useita erilaisia fysiologisia ja
  liikuntaan liittyviä mittauksia. Kuluttajat luottavat näihin mittauksiin ja kohdentavat esimerkiksi harjoitteluohjelmaansa
   tai levon määrää laitteiden mittauksien mukaisesti. Tässä Kandidaatintutkielmassa perehdytään siihen, kuinka tarkkoja
    nämä mittaukset ovat. Tutkielma käsittelee myös sitä, miten nämä mittaukset riippuvat ulkoisista muuttujista.
     Kandidaatintutkielmassa todetaan, että laitteet suoriutuvat mittauksista yleensä hyvin.
}

\abstract{%
Smartwatches and sports watches have become increasingly popular in recent years. These devices make several different physiological and exercise-related measurements.
 Consumers rely on these measurements and adjust their training programs or rest periods according to the device's measurements.
  This Bachelor's thesis explores how accurate these measurements are. The thesis also examines how these measurements depend on external variables.
   Thesis concludes that smart watches and sport watches usually perform well in measuring physiological and exercise-related measurements. 
}

\author{Roni Koskinen}
\contactinformation{\texttt{rpkoskin@student.jyu.fi}}
% use a separate \author command for each author, if there is more than one
\supervisor{Tytti Saksa}
% use a separate \supervisor command for each supervisor, if there
% is more than one

% you don't need this line in a thesis
% \type{Template and manual for a thesis document class}

\maketitle


Jyväskylä, \today

\bigskip

Roni Koskinen


\mainmatter

% \chapter{Introduction}
% Introduction

\chapter{Johdanto}
  Teknologian kehitys on johtanut liikunnan vähenemiseen, ja vapaa-ajan aktiviteetit on korvautunut aktiviteetteihin, joissa liikunta on vähäistä \parencite{petrusevski_interventions_2021}.
   Aktiivisuutta mittaavat laitteet ovat kuitenkin poikkeus tässä kehityksessä. Aktiivisuutta mittaavien laitteiden kerrotaan edistävän terveellisempiä elämäntapoja
    tekemällä aktiivisuusdatasta käyttäjälleen näkyvän \parencite{shin2015understanding}.
     Puettavat teknologiat ovat yleistyneet viimeisen kymmenen vuoden aikana tavallisten kuntoilijoiden käyttöön teknologian nopean kehityksen seurauksena.
      Teknologian nopea kehittyminen on laskenut laitteiden hintaa ja laitteiden fyysinen koko on pienentynyt. Nämä tekijät ovat mahdollistaneet laitteiden käytön ja hankkimisen suuremmalle joukolle ihmisiä.
       Teknologian kehittyminen on tehnyt laitteista aiempaa tarkempia ja parempia melkein kaikilla osa-alueilla. Niihin on myös kehitetty paljon uusia ominaisuuksia.
        Kellot muun muassa seuraavat unta, mittaavat sykettä, mittaavat kuljettua matkaa, noustua korkeutta ja happisaturaatiota.\\

  Tässä tutkielmassa tutkitaan, miten äly- ja urheilukellot suoriutuvat mittaustoiminnoistaan.
   Aihe on valittu kirjoittajan oman puettavien teknologioiden kiinnostuksen vuoksi. Monet käyttävät äly- tai urheilukelloa päivittäin, ja on mielenkiintoista tietää, mitä tutkimukset kertovat
    näiden teknologioiden mittauksien tarkkuudesta. \\
    
  Tutkimus tehdään kirjallisuuskatsauksena. Kirjallisuuskatsaus valittiin tutkimussuunnitelmaksi, sillä puettavista teknologioista on tehty paljon tutkimusta.
   Tutkimusstrategia on myös soveltuva aiheen laajuuden vuoksi. Aiheen laajuus vaatii useiden eri lähteiden yhdistelemistä yhdeksi kokonaisuudeksi. Yhden tutkimuksen tekeminen aiheesta on käytännössä mahdotonta,
    sillä tutkimus tutkimukseen tulisi sisällyttää lukuisia eri mittaustapoja ja mittaustilanteista. Tutkielmassa määritellään ensin, mitä tarkoittaa puettava teknologia
     sekä mitä tarkoitetaan äly- ja urheilukelloilla. Tämän jälkeen käydään läpi älykellojen mittauksista sykkeen, unen, matkan pituuden, korkeuden, askeleiden sekä
      happisaturaation mittauksen. Nämä on jaettu omiin kappaleisiinsa. Jokaisen kappaleen lopussa tiivistetään, suoriutuvatko laitteet mittauksesta hyvin.

   
\chapter{Puettavat teknologiat}

  Puettavat teknologiat tarkoittavat laitteita, jotka ovat suoraan tai irtonaisesti
   kiinni ihmisessä. Suoraan kiinni olevat laitteet ovat esimerkiksi äly- tai urheilukelloja.
    Irtonaiset puettavat teknologiat tarkoittavat yleensä puhelimia (\cite{godfrey2018z}.) Puettavia teknologioita voi olla
     monessa eri muodossa, ne voivat olla esimerkiksi koruja, lisävarusteita, lääketieteellisiä laitteita, vaatteita tai
      vaatteisiin liitettäviä laitteita \parencite{Yasar_what_wearable}.
    
  Puettavat teknologiat voidaan jakaa kahteen kategoriaan. Ensisijaisiin, eli laitteisiin jotka toimivat itsenäisesti
   ja yhdistävät muita laitteita toisiinsa \parencite{godfrey2018z}. Tällaisia ovat esimerkiksi äly -ja urheilukellot, puhelimet ja älysormukset.
    Toinen kategoria on toissijaiset, eli sellaiset laitteet, jotka tarvitsevat toimiakseen ensisijaisen laitteen johon ne lähettävät dataa. 
     Ensisijaiset laitteet mittaavat yleensä jotain tiettyä arvoa. Tällaisia ovat esimerkiksi rinnan ympärillä puettava sykevyö (\cite{godfrey2018z}.) \\

  \emph{Älykellot} ovat suosituin laite puettavien teknologioiden kategoriassa \parencite{godfrey2018z}.
   Älykellot tarkoittavat puhelimenkaltaisia laitteita, joita puetaan ranteessa \parencite{Gregersen_watch_2023}.
    Älykelloissa on myös yleensä joukko erilaisia antureita \parencite{rawassizadeh_wearables_2014}.\\

  \emph{Urheilukellot} tarkoittavat tietokonepohjaisi älykelloja, joka on kestävästi valmistettu, vedenkestävä ja siinä on ominaisuuksia kuten
    sykemittari sekä muita urheiluun, liikkumiseen ja terveyteen liittyviä seuranta ominaisuuksia \parencite{sportswatch_tfd}. Urheilukellojen ero älykelloon on siis se,
     että nämä ovat usein kestävämmin valmistettu, ja tarjoavat ensisijaisesti hyvinvoinnin ja urheilun seuraamiseen liittyviä ominaisuuksia. Erot ovat siis pienet, 
      ja esimerkiksi urheilukelloja valmistava Garmin käyttää sivuillaan kelloistaan "Smart Watch" termiä \parencite{garmin_site}. Tässä tutkielmassa
       käytetään termiä älykello ja urheilukello sen mukaan, mitä termiä tarkasteltavassa tutkimuksessa käytetään. Tutkielman kannalta näitä laitteita tarkastellaan kuitenkin
        yhteisenä kokonaisuutena.\\
  
  \chapter{Älykellojen mittaukset}
  Tässä luvussa käydään läpi älykellojen mittauksia. Tutkielmaan on valittu käsiteltäväksi sykkeen, matkan pituuden, korkeuden, unen, happisaturaation, sekä askelmittaamisen mittaamisen tarkkuudet.
   Älykellot suorittavat myös muita mittauksia, mutta niitä ei ole sisällytetty tähän tutkielmaan. Nämä mittaukset on valittu siksi, että ne on arvioitu
    älykellojen käytetyimmiksi ominaisuuksiksi\\


  \section{Syke}
  Sykkeen mittaamisen tarkkuuteen äly- ja urheilukelloissa löytyy tällä hetkellä eniten tutkimusta verrattuna muihin mitattaviin arvoihin.
   Kiinnostus sykkeen mittaamisen tarkkuuteen on siis ilmeinen. Kirjoittajan mielestä sykkeen mittaamisen tarkkuus onkin yksi tärkeimpiä älykellojen ominaisuuksia.
    Sykkeestä voidaan tehokkaasti seurata harjoituksen kuormittavuutta, ja sykkeen seuraaminen olisi hyvin vaikeaa ilman sykemittaria.
     Sykkeen mittaamiseen käytetään pääosin kahta erilaista menetelmää. Ensimmäinen menetelmä on elektrokardiografia (EKG, eng. ECG), jossa mitataan sydämen lyönnin
      synnyttämiä sähköaaltoja \parencite{noauthor_heart_nodate}. Käytän tässä tutkielmassa jatkossa mittaustavasta lyhennettä EKG.\\

       Toinen metodi taas on fotopletysmografinen (Optinen, FPG, eng. PPG), jossa mitataan valtimoiden supistumista ja laajenemista käyttäen infrapunavaloa \parencite{noauthor_heart_nodate}.
        Käytän tässä tutkielmassa jatkossa termiä ''optinen'' viitaten fotopletysmografiseen
         mittaukseen. Optisen mittauksen haittapuoli on siinä, että sykkeen mittaukseen vaikuttaa erilaiset yksilölliset tekijät \parencite{koerber_accuracy_2022}.
          Näitä tekijöitä käsitellään tutkielmassa myöhemmin.
          % Tällaisia eroja on todettu olevan esimerkiksi ihonvärin tummuus ja sukupuoli (\cite{shcherbina_accuracy_2017,hochstadt_continuous_2020}.)
          %  Yksilöllisten tekijöiden vaikutuksesta on kuitenkin saatu ristiriitaisia tutkimustuloksia \parencite{pasadyn_accuracy_2019}.
          %   Esimerkiksi \textcite{sanudo_pilot_2019, bent_investigating_2020} havaitsivat, että ihonvärin tummuudella ei ollut merkittävää vaikutusta sykemittauksen tarkkuuteen.
      
      
      Nelsonin ja Allenin (2019) mukaan älykellojen valmistajat käyttävät omia patentoituja algoritmeja optisen sensorin
       signaalien muuntamiseksi sykearvoiksi. Tämä vaikuttaa myös siihen, miksi eri valmistajien kelloilla saattaa olla isojakin eroja
        sykemittauksen tarkkuudessa. Optisen mittauksen hyöty on myös, että valon avulla voidaan arvioida veren happipitoisuuden määrää \parencite{noauthor_heart_nodate}.
         Tämä onkin ollut viimevuosina suosittu tutkimuksen kohde.\\


      
  EKG mittaamista käytetään yleensä rinnan ympärillä puettavissa sykevöissä, kun taas optista tekniikkaa ranteessa puettavissa laitteissa \parencite{noauthor_heart_nodate}.
   Optisten sensoreiden tarkkuutta tutkittaesa vertailukohtana käytetään usein Polarin EKG-mittausta hyödyntäviä sykevöitä, sillä niiden on todettu olevan hyvin lähellä
    lääketieteessä käytettävien EKG-laitteen mittauksia. Varsinkin alhaisen, keskisuuren ja jopa suuren harjoitteluintensiteetin aikana
     (\cite{gilgen-ammann_rr_2019, nelson_accuracy_2019}.) \\


   \textcite{wang_accuracy_2017} tutkivat miten Polar H7-sykevyö, sekä neljän erilaisen älykellon sykemittaukset vertautuivat lääketieteessä käytettävään
    EKG-laitteen sykemittaukseen. Osallistujien sykettä mitattiin elektrodiagrammin avulla siten,
     että heillä oli neljä johtoa kiinnitettyinä elektrodeihin jokaisessa raajassa. Polar H7-sykevyön sekä kaksi satunnaista älykelloa molemmissa ranteissa.
      Osallistujat juoksivat matolla 2, 3, 4, 5 ja 6 mailia tunnissa nopeuksilla, jokaisella nopeudella kolmen minuutin ajan \parencite{wang_accuracy_2017}.
       Tutkimuksen mukaan laitteet korreloivat elektrodiagrammin mittauksiin kaavion \ref{table:wang} mukaisesti.\\


  \begin{table}[H]
    \begin{center}
      \begin{tabular}{||c c||} 
       \hline
       Älykello & Korrelaatio\\
       \hline\hline
       Polar H7 & 0,99 (0,987-0,991)\\
       \hline
      Apple Watch & 0,91 (0,884-0,929)\\
      \hline
      Mio Fuse & 0,91 (0,882-0,929)\\
      \hline
      Fitbit Charge HR & 0,84 (0,791-0,872)\\
      \hline
      Basis Peak & 0,83 (0,779-0,865)\\[0.5ex]
      \hline
      \end{tabular}
    \caption{Älykellojen sykemittauksen tarkkuuden korrelaatio elektrodiagrammin sykemittaukseen \parencite{wang_accuracy_2017}}
    \label{table:wang}
      \end{center}
    \end{table}

  Laitteet olivat levossa tarkempia kuin intensiivisen harjoittelun aikana \parencite{wang_accuracy_2017}. Tämän näkee kuviosta vertaamalla korrelaation ylärajaa sen alarajaan.
   Tähän saatta olla syynä se, että harjoittelun aikana laite heiluu, ja optinen mittaus saattaa häiriintyä. Tutkielmassa käsitellään vielä myöhemmin tekijöitä, jotka voivat vaikuttaa mittausten tarkkuuteen. 
    Polar H7 sykevyön mittasi siis sykettä tutkimuksen mukaan hyvin tarkasti, korreloiden 0,99 EKG-laitteen mittauksiin \parencite{wang_accuracy_2017}.
     Kuten taulukosta \ref{table:wang} huomataan, kahden kellon mittausten korrelaatio oli keskimäärin jopa alle 0,85, joten niiden sykemittaukset eivät olleet kovin tarkkoja.\\
  
  \textcite{nelson_accuracy_2019} tutkivat miten tarkasti Apple Watch 3, sekä Fitbit Charge 2 mittasivat sykettä jokapäiväisessä elämässä.
   Tutkimuksessa osallistuja pitivät kelloa 24 tunnin ajan ranteessaan eläen normaalia arkea, ja mittauksia verrattiin
    EKG-vertailulaitteen tuloksiin \parencite{nelson_accuracy_2019}. Tulokset on esitetty taulukossa \ref{table:nelson}.\\
    
    \begin{table}[H]
     \begin{center}
      \begin{tabular}{||c c c||}   
       \hline
       Aktiviteetti & Apple Watch 3 & Fitbit Charge 2\\ [0.5ex] 
       \hline\hline
       24H keskiverto & 5,86 & 5,96\\ 
       \hline
       Istuminen & 7,21 & 6,93\\ 
       \hline
       Juoksu & 3,01 & 9,88 \\
       \hline
       Kävely & 4,64 & 9,21\\
       \hline
       Nukkuminen & 3,12 & 3,36\\[1ex] 
       \hline
      \end{tabular}
    \caption{Apple Watch 3 ja Fitbit Charge 2 sykemittauksen ero EKG mittarin tuloksiin. Arvot kuvaavat suhteellisen virheen keskimääräistä arvoa (MAPE). Lähde: \textcite{nelson_accuracy_2019}}
    \label{table:nelson}
     \end{center} 
    \end{table}

  Tulokset eroavat \textcite{wang_accuracy_2017} tutkimuksen tuloksiin siten, että Apple Watchin tarkkuus parani juostessa, verrattuna siihen mitä se oli istuessa. Wangin ym. (2017) tutkimuksessa
   mittausten tarkkuus heikkeni intensiivisemmän harjoittelun aikana. Tämä voi johtua monesta eri syystä, sillä sykkeen mittauksen tarkkuuteen vaikuttaa yksilölliset tekijät \parencite{koerber_accuracy_2022,pasadyn_accuracy_2019,hochstadt_continuous_2020}.
    Tutkimuksissa käytettiin myös erilaisia laitteita, joiden tarkkuus voi olla eri. \\

  \textcite{pasadyn_accuracy_2019} tutkivat neljän eri äly- sekä urheilukellon tarkkuutta juoksumatolla juosten kuudella eri nopeudella. Tässäkin tutkimuksessa tuloksia verrattiin
   Polar H7 sykevyön mittaustuloksiin. Tutkimuksessa Apple Watch 3:n korrelaatio EKG-mittauksiin oli 0,96, Fitbit iconic, Garmin Vivosmart HR ja Tom Tom Spark 3
    kellojen korrelaatio ECG tuloksiin oli kaikilla sama 0,89 \parencite{pasadyn_accuracy_2019}. Apple Watch 3 oli siis huomattavasti tarkempi kello sykkeen mittaukseen,
     kuin muut tutkimuksessa käytetyt laitteet. Tutkimuksen mukaan laitteiden sykkeen mittaamisen tarkkuus laski
      harjoitteluin intensiteetin noustessa \parencite{pasadyn_accuracy_2019}. Tämä tulos on yhtenäinen Wangin ym. (2017) tutkimuksen kanssa.\\
  
  
   Sykkeenmittauksen voidaan siis sanoa olevan useimmiten tarkkaa.
    Tarkkuudessa on kuitenkin eroja kellojen välillä, eivätkä laitteet kykene täysin tarkkoihin sykkeen mittauksiin
     (\cite{pasadyn_accuracy_2019,wang_accuracy_2017,nelson_accuracy_2019}.) Tarkkuutta arvioidessa on myös hyvä ottaa huomioon, että mittauksiin saattaa vaikuttaa
      yksilölliset tekijät, kuten ihonväri tai sukupuoli \parencite{shcherbina_accuracy_2017,hochstadt_continuous_2020}. Tämä johtuu älykelloissa käytettävän optisesta sensorista,
       sillä valon avulla tehty mittaus saattaa häiriintyä, jos valo ei esimerkiksi läpäisekään ihoa tarpeeksi hyvin \parencite{koerber_accuracy_2022}.
       Tutkimustulokset yksilöllisten tekijöiden vaikutuksista ovat kuitenkin ristiriitaisia \parencite{koerber_accuracy_2022, pasadyn_accuracy_2019}.
        Yksi syy tähän saattaa olla, että optinen mittausteknologia on kehittynyt niin, että esimerkiksi ihonvärin tummuus ei enää vaikuta mittaustuloksiin.
         \textcite{koerber_accuracy_2022} esittävät, että tähän voisi olla syynä vihreän värin käyttäminen optisissa sensoreissa punaisen värin sijaan.
          Vihreän valon 520 aallonpituuden on ainakin todettu läpäisevän ihon paremmin, kuin muiden aallonpituuksien \parencite{fallow_influence_2013}.
        
       

  \section{Uni}
  Riittävä unen saanti on tärkeä terveyden ja hyvinvoinnin ylläpitämiseksi. Watsonin (2017) mukaan unella on merkitystä niin fyysiseen kehitykseen,
   emootioiden hallintaan, kognitiiviseen suorituskykyyn ja muutenkin elämänlaatuun. 
    \textcite{watson_sleep_2017} mainitsee myös, että urheilijat arvioivat unensa laatua ja kestoa heikosti. Mahdollisuus unen mittaukseen
     on siis varsinkin urheilijoille hyödyllinen ominaisuus. Unen seurannan tarkkuutta tutkitaan vertaamalla tuloksia ''gold standard''
      unen mittausmenetelmään, eli polygrafiamittaukseen (PSG) \parencite{de_zambotti_measures_2016, rundo_chapter_2019, miller_validation_2022}.\\
  
  Unen laatu on vaikea määritellä objektiivisesti. Unen laatua voidaan tarkastella subjektiiville tasolla, sillä unen laadulla tarkoitetaan subjektiivista kokemusta siitä, kuinka hyvin on henkilö on nukkunut.
   Unen laatua tarkastellaan siten, kokeeko henkilö olonsa väsyneeksi nukutun yön jälkeen vai ei. Levänneisyys on
    unen laadun ja määrän yhdistelmä, eikä toinen voi korvata toista. Unta arvioitaessa tulee siis ottaa molemmat asiat huomioon.
     Unen laadun mittaamiseen on useita erilaisia keinoja (\cite{kohyama_which_2021}.)
      Jotkin näistä menetelmistä käyttävät unen vaiheita mittaamaan unen laatua, esimerkiksi
       NREM-unen määrän on ainakin todettu olevan yhteydessä arvioituun unen laatuun (\cite{krystal_measuring_2008}.)
        Jotkut älykellot antavat arvion nukutun unen laadusta, mutta näiden arvioiden tarkkuudesta ei ole tehty tutkimusta.\\
  
  Valmistajat eivät usein kerro, millaisilla algoritmeilla he arvioivat unen vaiheita tai unen laatua.
   Tiedetään kuitenkin, että älykellot käyttävät unen vaiheiden arviointiin ainakin kiihdytinsensoria, joka mittaa liikettä unen aikana. Liikkeen määrästä voidaan
    päätellä, onko henkilö kevyen unen vaiheessa (\cite{grifantini_hows_2014,de_zambotti_measures_2016}.)
     Tiedetään myös, että kehon autonomisista toiminnoista, kuten sykkeestä tai ihon lämpötilasta voidaan tehdä päätelmiä, missä
      unen vaiheessa henkilö on \parencite{de_zambotti_measures_2016, fonseca_sleep_2015}. Ei voida kuitenkaan varmuudella sanoa, käyttävätkö älykellot näitä mittauksia unen arviointiin.
       Seuraavaksi käyn läpi tutkimuksia älykellojen unenseurannan tarkkuudesta.\\
    
       

   Zambotti ym. (2016) mukaan Fitbit -älykello yliarvioi unta keskimäärin
    8 minuuttia pidemmäksi, aliarvioi hereilläoloaikaa nukahtamisen jälkeen 5,6 minuuttia ja yliarvioivat unen tehokkuutta 1,8\%.
     Fitbit-kello tunnisti unen alkamisen ja unitilan hyvin tarkasti.
      Tutkimuksen mukaan kello ei kuitenkaan tunnistanut hereilläoloaikaa kovin tarkasti nukahtamisen jälkeen.\\
  
  \textcite{liang2018validity} tutkivat Fitbit Charge 2 -kellon unen seurannan tarkkutta ja saivat samankaltaisia tuloksia
   kuin \textcite{de_zambotti_measures_2016}. Liangin ja Chapa Martellin (2016) mukaan Fitbit Charge 2 -kello tunnisti unen keston
    ja unen tehokkuuden tarkasti. Hereilläolon tunnistaminen nukahtamisen jälkeen on kuitenkin
     epätarkkaa. Tämä tutkimuksen tulos on yhtenäinen \textcite{de_zambotti_measures_2016} tutkimuksen tuloksien kanssa.
      Tulokseksi saatiin myös, että unen vaiheiden, kuten kevyen, REM sekä syvän unen tunnistaminen oli kellolle haastavaa.\\
  
  \textcite{chinoy_performance_2022} tutkimuksen mukaan Fitbit Inspire HR:n sekä Polar Vantage V Titanin unen seurannan tarkkuus oli 
    samankaltainen edellisten tutkimusten tulosten kanssa. Kellot mittasivat unen keston tarkasti,
     mutta hereilläoloajan mittaus nukahtamisen jälkeen oli epätarkkaa \parencite{chinoy_performance_2022}.
      Kellot arvioivat myös unen vaiheita epätarkasti \parencite{chinoy_performance_2022}.\\
  
  Unen määrän mittaamisen on todettu olevan suhteellisen tarkkaa lukuisissa tutkimuksissa
   \parencite{de_zambotti_measures_2016,liang2018validity,chinoy_performance_2022,miller_validation_2022}. Voidaan siis kokoavasti todeta, että kellot suoriutuvat
    pääasiassa hyvin unen määrän mittauksesta. Voidaan myös todeta, että kellot eivät kuitenkaan suoriudu tarkasti unen vaiheiden erottelusta
     \parencite{chinoy_performance_2022,de_zambotti_measures_2016,liang2018validity}.
       
         
  \section{Paikanninjärjestelmä}
  Niin ammatti- kuin amatööri juoksijat nojautuvat pitkälti urheilukellon GNSS:n antamaan lukemaan juostun matkan pituudesta.
   Urheilukellojen sijainnin tarkkuuden mittaamisesta on kuitenkin tehty vähän tieteellistä tutkimusta \parencite{gilgen-ammann_accuracy_2020}.
    GNSS-mittaus on käytössä useassa lajissa, ja olisi tärkeä saada tietoa siitä, ovatko mittaukset luotettavia. Seuraavaksi käydään läpi tutkimuksia
     paikanninjärjestelmän tarkkuudesta äly- ja urheilukelloissa.\\

  Äly- ja urheilukellot käyttävät sijainnin määrittämiseen Global navigation satellite system (GNSS) vastaanottimia \parencite{gilgen-ammann_accuracy_2020}. 
  Sijaintimittausten avulla kello laskee urheilusuorituksen pituuden ja nopeuden. GNSS:ään lukeutuu Euroopan Galileo järjestelmä, Yhdysvaltojen Global Navigation System (GPS),
   Venäjän Glosnass sekä kiinan BeiDou satelliittipaikannusjärjestelmät \parencite{hofmann2007gnss}.
    Kelloissa on eroa siinä, mitä satelliittijärjestelmiä ne voivat käyttää \parencite{ammann_accuracy_2016}.
     Joissain tutkimuksissa puhutaan GPS-paikantimen tarkkuudesta, toisissa GLOSNASS:ista, ja toisissa taas viitataan yleisesti GNSS:ään. Puhuttaessa GNSS:stä tai GPS:stä
      tarkoitetaan kuitenkin samaa älykellon paikannin sirua, ainoastaan satelliitit joihin yhdistetään muuttuvat.\\

  GPS-mittaamisen tarkkuuteen vaikuttaa lukuisia ulkopuolisia seikkoja. Tarkkuuteen vaikuttaa esimerkiksi laitteen päivitysfrekvenssi.
   Mitä suurempi frekvenssi sen tarkempi sensori on (\cite{cummins_global_2013}.) Tarkkuuteen vaikuttaa myös
    se, onko GPS-sirun lähellä korkeita rakennuksia ja onko taivas selkeä \parencite{baranski_enhancing_2012}. \\

  \textcite{gilgen-ammann_accuracy_2020} tutkivat Applen, Coroksen, Polarin ja Suunnon urheilukellojen GPS-mittauksen tarkkuutta urheilusuorituksissa, jotka olivat
    pituudeltaan 400-4000 metriä. Lajeihin kuuluivat juoksu, kävely sekä pyöräily. Keskimäärin urheilukellojen GPS-mittauksissa todettiin olevan 3-6\% virhe.
    Tutkijoiden mukaan urheilukellojen matkan pituuden mittaukset olivat kohtuullisia tai hyviä (\cite{gilgen-ammann_accuracy_2020}.)\\

  \textcite{johansson_accuracy_2020} tutkivat urheilukellojen GPS:n tarkkuutta ultramaratonjuoksussa. Tutkimuksessa juoksureitin pituus oli 56 kilometriä ja reitti sisälsi 800 nousumetriä.
   Tutkimuksessa oli mukana useita eri äly- ja urheilukelloja, sekä puhelimia.
    GPS-laitteiden tarkkuus oli tarkimmillaan $\SI{0,6 \pm 0,3}{\percent}$ ja epätarkimmillaan $ \SI{1,6 \pm 0,9}{\percent} $. (Mediaani $\pm$ IQR) (Kvartiiliväli, eng. Interquartile range).
     Tässä statistiikassa oli mukana myös puhelimet, mutta 
      tulokset ovat silti huomattavasti tarkempia, kuin tutkimuksessa jonka suoritti \textcite{gilgen-ammann_accuracy_2020} suorittamassa tutkimuksessa. Puhelimet olivat tutkimuksen epätarkimpia laitteita \parencite{johansson_accuracy_2020}.
       Tulokset ovat erikoisia siksi, että \textcite{johansson_accuracy_2020} tutkimusasetelma on suoritettu vapaammissa olosuhteissa. Reitillä on myös ollut puita ja vuoria, jotka saattaavat häiritä GPS-signaalia.
        Johanssonin ym. (2020) mukaan urheilukellot ovat pätevä tapa mitata juostua matkaa.\\

  Voidaan todeta, että äly- ja urheilukellot suoriutuvat pääosin hyvin juoksu-, kävely- sekä pyöräilysuoritusten pituuden mittauksessa \parencite{gilgen-ammann_accuracy_2020,johansson_accuracy_2020}.
   Kellot käyttävät nopeuden mittaamiseen paikanninjärjestelmän tuloksia, joten voidaan todeta kellojen suoriutuvan nopeuden mittauksestakin hyvin. On kuitenkin huomioitava,
    että GPS-mittauksen tarkkuuteen saattaa vaikuttaa esimerkiksi ympäristössä olevat korkeat rakennukset tai taivaan pilvisyys \parencite{baranski_enhancing_2012}.

  \section{Korkeuden mittaus}
    Noustut metrit harjoittelun aikana on tärkeä osa urheilijan kokonaistyömäärää. Tämän vuoksi älykellojen korkeuden mittaamisen tarkkuutta on hyvä tutkia (\cite{ammann_accuracy_2016}.)
     Äly- ja urheilukellot käyttävät korkeuden mittaamiseen barometristä sensoria tai GPS-paikantimen antamaa informaatiota \parencite{ammann_accuracy_2016}.
      Barometrinen sensori arvioi korkeuden mittaamalla ilmanpainetta \parencite{aroganam2019review}.
       On myös mahdollista, että kello käyttää GPS:n ja barometrin kombinaatiota \parencite{aroganam2019review}. Tällä tavalla kello yrittää korjata barometrin mahdolliset ulkoisista seikoista
        johtuvat mittausvirheet \parencite{aroganam2019review}.\\
        
        Barometriä on aiemmin käytetty sään ennustamiseen \parencite{manivannan_challenges_2020}.
         Barometrinen sensori onkin siis herkkä säätilojen muutoksille, jonka vuoksi sääolosuhteet tulee ottaa huomioon laitteen tarkkuutta tutkittaessa \parencite{manivannan_challenges_2020, ammann_accuracy_2016}.
          Barometrin tarkkuuteen voi vaikuttaa myös ympäristö, korkeus ja sensorin tarkkuus (\cite{manivannan_challenges_2020}.)
           Barometrien tarkkuudesta äly- tai urheilukelloissa on kuitenkin tehty suppeasti tutkimusta.\\
           
    \textcite{ammann_accuracy_2016} tutkivat älykellojen korkeusmittauksen tarkkuutta erilaisilla juoksureiteillä. Reiti erosivat siinä, kuinka paljon niillä oli nousumetrejä.
     Ensimmäinen reitti oli tasainen, toinen sellainen, jossa nousua oli 90 metriä, ja kolmas reitti oli silmukka, jossa yhdellä kerralla oli nousua 30 metriä.
      Tätä reittiä kutsuttiin ''mäkiseksi reitiksi''.
       Testaamiseen käytettiin Garmin ForerunnerXT, Polar RS800XC, sekä Suunto Ambit 2 -urheilukelloja. Näistä kelloista Garmin ja
        Suunto käyttivät korkeuden mittaamiseen barometrin sekä GPS yhdistelmää.
         Kellojen kalibroitiin ennen jokaista testiä vastaamaan kyseistä korkeutta merenpinnan yläpuolella, jossa testaajat olivat (\cite{ammann_accuracy_2016}.)\\

    Ammannin ym. (2016) mukaan kellot aliarvioivat mäkisellä reitillä nousumetrejä 3,3\%-9,8\%. Tasaisella reitillä mittaukset olivat suhteellisen tarkkoja.
     Mittaukset erosivat todellisiin nousumetreihin tasaisella reitillä 0,0\%-0,4\%. Mittaukset olivat siis todella tarkkoja tasaisella reitillä.
      Tutkijat toteavat, että urheilukellot olivat melko tarkkoja nousumetrien mittaamisessa. Mittaukset eivät olleet kuitenkaan
       täysin tarkkoja, ja eri valmistajien kellojen välillä oli tarkkuus eroja (\cite{ammann_accuracy_2016}.)\\
       
    Tutkimuksen tuloksia tarkasteltaessa on otettava huomioon, että tutkimuksessa käytettävät kellot oli julkaistu vuosina 2008, 2012 ja 2013. Tekniikka on voinut siis kehittyä tähän mennessä. 
     \textcite{ammann_accuracy_2016} ottivat tutkimuksessa huomioon ulkopuoliset vaikutukset mahdollisimman hyvin, mutta toteavat, että mittauksiin
      saattoi vaikuttaa esimerkiksi sääolosuhteiden muutos, tai GPS-signaalin vaihteleva vahvuus.\\

  \section{Askeleet}
  Aktiviteetit kuten käveleminen muodostavat suurimman osan terveydelle ja hyvinvoinnille suunnitelluista fyysisestä aktiviteetista \parencite{gaz_determining_2018}.
   Tämän vuoksi aktiivisuutta mittaavien laitteiden askelmittarin tarkkuus on tärkeä määrittää.
    Gazin ym. (2018) mukaan tutkimuksessa käytettyjen älykellojen askelmittarin tarkkuus oli kohtuullisen hyvä. Askeleiden mittaamisen tarkkuus
     vaihteli kuitenkin riippuen laitteesta ja kävelytilanteesta. Askelmittari oli tarkempi juoksumatolla, kuin
      maalla kävellessä (\cite{gaz_determining_2018}.)\\
  
  \textcite{ahanathapillai_preliminary_2015} saivat samanlaisia tuloksia askelmittauksen tarkkuudesta Android älykelloissa, kuin \textcite{gaz_determining_2018}.
   Ahanathapillain ym. (2015) mukaan Android käyttöjärjestelmää käyttävät älykellot askelmittaus oli tarkkaa.
    Normaalisti kävellessä älykellon keskimääräinen virhearvio oli 1,25\%, kun älykellon päivitystaajuus oli 50 Hz.
     Tarkkuus kuitenkin kärsi kävellessä portaissa tai päivitystaajuuden laskiessa (\cite{ahanathapillai_preliminary_2015}.)\\
  
  Voidaan siis sanoa, että älykellojen tarkkuus mitata askeleita yleisimmissä olosuhteissa on suhteellisen tarkka
   \parencite{gaz_determining_2018,ahanathapillai_preliminary_2015}. On kuitenkin otettava huomioon, että mittareiden
    tarkkuudessa on eroja, jotka riippuvat olosuhteista ja älykellon mallista \parencite{gaz_determining_2018}.
     Tutkimusta äly -ja urheilukellojen askelmittauksesta ei kuitenkaan ole kovin paljoa.
      Lisää tutkimusta laitteiden askelmittauksen tarkkuudesta olisi hyvä tehdä.
       Varsinkin siksi, että teknologia kehittyy jatkuvasti eteenpäin ja laitteista tulee tarkempia.
  

  \section{Happisaturaation mittaus}
  Happisaturaation (SpO2) mittaus on tullut älykelloihin vuoden 2021 aikana \parencite{zhang_can_2022}. Kyseessä on siis suhteellisen uusi ominaisuus.
   SpO2 mittauksien tarkkuudesta älykelloissa ei vahvistettu \parencite{zhang_can_2022}. Tämä johtuu tehtyjen tutkimusten vähyydestä.
    SpO2 mittaus toimii älykelloissa optisesti \parencite{windisch_accuracy_2023}, niin kuin sykkeen mittaus.\\
    
  \textcite{pipek_comparison_2021} mukaan Apple Watch Series 6 -älykellon SpO2-mittauksen korrelaatio vertailulaitteeseen oli 0,81.
   Apple Watchin mittauksilla olivat taipumus olla hieman korkeampia, kuin vertailulaitteen arvoilla \parencite{pipek_comparison_2021}.
    Älykellon SpO2-mittauksen tarkkuus korrelaatio vertailulitteen mittauksiin oli 0,81.
     \textcite{pipek_comparison_2021} mukaan Apple Watch 6 älykello on luotettava väline SpO2-mittaukseen kontrolloidusssa ympäristössä.
      Mittauksia verrattiin kahteen kaupallisesti saatavilla olevaan pulssioksimetriin.
       Tutkimuksissa ei mainittu, mitä pulssioksimetriä vertailukohtana käytettiin, tai oliko vertailulaitteen tarkkuus varmistettu aiemmilla tutkimuksilla.
   
  \textcite{patz_accuracy_2023} saivat samankaltaisia tuloksia Apple Watchin pulssioksimetrin tarkkuudesta kuin \textcite{pipek_comparison_2021}. Mittauksista 85\%
   oli oikein mitattaessa lapsilta ja 85\% aikuisilta mitattaessa. Pätzin ym. (2023) mukaan Apple Watch 6 mittaukset eivät ole
    lääketieteellisten standardien tasolla. Tarkkuus on kuitenkin hyvä antamaan arviota happisaturaation määrästä.
     Siihen ei kuitenkaan tule luottaa täysin.\\

    Voidaan todeta, että SpO2-mittaus Apple Watch -älykellolla on hyvä. Laite ei kuitenkaan ole niin tarkka, että sitä voitaisiin käyttää
     lääketieteessä. Mittaukset ovat suuntaa antavia. Edellä olevat tutkimukset oltiin suoritettu kontrolloiduissa olosuhteissa, joten niiden vaikutusta ei ole vahvistettu tutkimuksilla.
      Koska SpO2 mittaus toimii optisesti niin kuin sykkenkin mittaus, voisi päätellä, että SpO2 mittaukseen pätee samat rajoitteet kuin sykkeenkin mittaukseen. \\

      
    
  
  % \chapter{Älykellojen vaikutus hyvinvointiin}
  % Mitä vaikutuksia älykelloilla on ihmisten hyvinvointikäyttäytymiseen. Auttaako laitteet lisäämään liikuntaa, enemmän unta tai terveellisempiä elämäntapoja ylipäätään?
  % \section{Positiiviset}
  % \section{Negatiiviset vaikutukset}
  
\chapter{Johtopäätökset}

Tässä tutkielmassa tuotiin esiin tutkimustuloksia äly -ja urheilukellojen mittausten tarkkuudesta.
 Älykellojen mittaukset ovat usein riippuvaisia erilaisista ulkoisista seikoista. Näiden seikkojen merkittävyydestä on kuitenkin ristiriitaisia tutkimustuloksia.
  Esimerkiksi sykkeenmittauksen tarkkuuden riippuvuudesta sukupuolesta tai ihonvärin tummuudesta on saatu tutkimustuloksia
   osoittaen, että näillä tekijöillä ei ole merkitystä, sekä että niillä on merkitystä. GPS, sekä korkeuden mittaamiseen tarkkuuteen tiedetään myös vaikuttavan
    ulkoiset tekijät, kuten sääolosuhteet tai ympäristössä olevat korkeat rakennukset tai puut. Näiden vaikutuksesta ei kuitenkaan ole paljoa tutkimustietoa.
     On myös ilmeistä, että mittausten tarkkuuteen vaikuttaa kellon malli. Jotkin algoritmit ja sirut ovat tarkempia kuin toiset.\\

Älykellot suoriutuivat matkan ja korkeuden mittauksista hyvin. Tutkimuksissa saatiin hieman erilaisia arvoja tarkkuuksista, mutta kaikissa
 todettiin matkan mittaamisen kuitenkin olevan tarkkaa. Tutkimusdataa korkeudenmittauksista oli kuitenkin vähän. 3,3\%- 9,8\%
  Sykkeenmittauksen voidaan todeta olevan äly- ja urheilukelloissa suhteellisen tarkkaa. Tarkkuus riippuu kuitenkin kellon mallista.
   Esimerkiksi Apple Watchin korrelaatio EKG-mittarin tuloksiin oli 0,91, kun taas Basis Peak kellon korrelaatio vertailulaitteen mittauksiin
    oli ainoastaan 0,83 (\cite{wang_accuracy_2017}.) Joissain tutkimuksissa Apple Watchin korrelaatio vertailulaitteen tuloksiin oli vieläkin korkeampi.
     Mittaukset eivät ole siis täysin tarkkoja, mutta riittäviä suurimmalle osalle käyttäjistä. Sykkeen mittauksen tarkkuuteen vaikuttaa myös
      aktiviteetti. Sykkeen mittaus ovat tarkempia \\

Unen mittauksen tarkkuus riippuu siitä, mitä arvoja halutaan mitata. Älykellot mittasivat Unen pituuden tarkkuutta hyvin, mutta unen vaiheiden mittaus oli epätarkkaa.
 Löydökset todettiin olevan samankaltaisia useammalla eri kellomallilla. 
 Unen vaiheiden erottelu ei kelloilta kuitenkaan onnistunut tarkasti. Unen vaiheet ovat yhteydessä unen laatuun \parencite{krystal_measuring_2008},
  ja unen laatu on erittäin tärkeä arvo arvioidessa unta sen pituuden ohella \parencite{kohyama_which_2021}. Unen vaiheiden
   tarkempi arviointi olisi siis tärkeää unen arvioimisen laadukkuuden kannalta.\\
  
SpO2 mittaukset ovat Apple Watch kellossa suuntaa antavia.
 \textcite{patz_accuracy_2023} mukaan mittauksista noin 85\% oli oikein. \textcite{pipek_comparison_2021} mukaan mittausten korrelaatio vertailulaitteen tuloksiin oli 0,81.
   Tulokset voivat siis antaa arviota SpO2 arvosta, mutta niitä ne eivät ole täysin tarkkoja. Olosuhteiden vaikutuksesta mittauksiin ei ole varmuutta.\\

Tässä kandidaatintutkielmassa on vahvuutena se, että tutkielman laatimiseen on hyödynnetty useita eri tutkimuksia, ja yhdistelty näiden tuloksia toisiinsa.
 Tutkimukset mittauksien tarkkuuksista ovat olleet myös suhteellisen tuoreita, sillä ne on julkaistu vuonna 2016 tai sen jälkeen.
  Heikkouksina tutkielmassa on, ettei tutkimuksia ei ole saatavilla uusimmista äly- tai urheilukelloista. Tekniikka kehittyy nopeasti,
  melkein joka vuosi tulee uusia malleja, joten uusista malleista ei ole ehditty tuottaa tutkimusta.
     Toinen heikkous on tutkielman teknologinen rajaus, joka on hyvin on laaja. Tutkielmaan otettiin mukaan monen eri valmistajan laitteita, ja niissä
      on keskenään suuriakin tarkkuus eroja. Rajauksen vuoksi tutkielma voitaisiin tehdä ainoastaan käyttäen tiettyä äly- tai urheilukello mallia.
       Tutkielma toimii hyvin tiivistelmänä siitä, missä vaiheessa modernien älykellojen tarkkuus on tällä hetkellä. Aihe on ajankohtainen
        ja tärkeä yleisen tietoisuuden lisäämiseksi siitä, että laitteet ovat erehtyväisiä, eikä niiden tarkkuus ole läheskään aina täydellinen.


% The last chapter of a thesis is the Conclusion (some authors use
% Conculsions, instead).  Keep it short, and discuss what one can
% conclude about the thesis statement or research question given in the
% Introduction, in light of all that has been written in the thesis.
% The Conclusion is also the place to discuss any limitations and
% weaknesses of the thesis (especially those that cast doubt on the
% reliabliity of the results given in the thesis), if they have not been
% already discussed, for example in a Discussion chapter.  It is also
% customary to state, what further research might be beneficial in light
% of this thesis.

% If the Conclusion threatens to become too long, it is a good idea to
% split the interpretation of the results into its own chapter, often
% called Discussion, making Conclusion short and sweet.

% After Conclusion, there is the bibliography, indicated by the
% \string\printbibliography\

\printbibliography



\end{document}
