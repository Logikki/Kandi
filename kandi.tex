\documentclass[utf8,bachelor,finnish]{bachelor}
% If you are writing a Master's Thesis, use the following instead:
%\documentclass[utf8,english]{gradu3}
% overleaf pohja : https://www.overleaf.com/project/63e5ffa6c6c53ac541c21a89
\usepackage{graphicx} % for including pictures

\usepackage{amsmath} % useful for math (optional)
\usepackage[
  separate-uncertainty = true,
  multi-part-units = repeat
]{siunitx}

\usepackage{booktabs} % good for beautiful tables

% NOTE: This must be the last \usepackage in the whole document!
\usepackage[bookmarksopen,bookmarksnumbered,linktocpage]{hyperref}

\addbibresource{references.bib} % The file name of your bibliography database

\begin{document}

\title{Miten äly- ja urheilukellot suoriutuvat niille sunnitelluista tehtävistä}
\translatedtitle{\LaTeX-tutkielmapohjan {gradu3} käyttö}
\studyline{Mathematical Information Technology}
\avainsanat{%
  \LaTeX,
  {gradu3},
  pro gradu -tutkielmat,
  kandidaatintutkielmat,
  käyttöohje}
\keywords{\LaTeX, {gradu3}, Master's Theses, Bachelor's Theses, user's guide}
\tiivistelma{%
 tiivistelmä sama kuin eng abstract
}
\abstract{%
  Tiivistelmä in english
}

\author{Roni Koskinen}
\contactinformation{\texttt{rpkoskin@student.jyu.fi}}
% use a separate \author command for each author, if there is more than one
\supervisor{Tytti Saksa}
% use a separate \supervisor command for each supervisor, if there
% is more than one

 % you don't need this line in a thesis
\type{Template and manual for a thesis document class}

\maketitle


Jyväskylä, \today

\bigskip

The Author


\mainmatter

% \chapter{Introduction}
% Introduction

\chapter{Johdanto}
  Teknologian kehitys on johtanut liikunnan vähenemiseen ja vapaa-ajan aktiviteetit on korvautunut aktiviteetteihin, joissa liikunta on vähäistä \parencite{petrusevski_interventions_2021}.
   Aktiivisuutta mittaavat laitteet ovat kuitenkin poikkeus. Aktiivisuutta mittaavaavien laitteiden kerrotaan edistävän terveellisempiä elämäntapoja, tekemällä aktiivisuusdatasta näkyvän, kerrotaan Shinin ym. artikkelissa (2015) \parencite{shin2015understanding}.
  Puettavat teknologiat ovat yleistyneet viimeisen kymmenen vuoden aikana tavallisten kuntoilijoiden käyttöön, sillä teknologia on kehittynyt nopeasti.
    Teknologian nopea kehittyminen on tuonut hinnat alas, ja laitteiden fyysinen koko on pienentynyt, joten niiden käyttö ja hankkiminen on mahdollista suuremmalle joukolle ihmisiä.
     Teknologian kehittyminen on tehnyt laitteista aiempaa tarkempia ja parempia melkein kaikilla osa-alueilla. Niihin on myös kehitetty paljon uusia ominaisuuksia.
      Kellot muunmuassa seuraavat unta, mittaavat sykettä, mittaavat kuljettua matkaa sekä noustua korkeutta.\\

  Kanditutkielmassani tutkitaan sitä, miten puettavat hyvinvointiteknologiat suoriutuvat ja soveltuvat siihen, mitä toimintoja ne on suunniteltu tekemään.
   Valitsin aiheen siksi, että puettavat teknologiat ovat mielestäni kiinnostavia. Käytän päivittäisessä elämässäni älykelloa, ja haluaisin tietää mitä tutkimukset kertovat
    siitä, miten hyvin nämä teknologiat soveltuvat ja suoriutuvat tarkoituksestaan hyvinvointia edistävinä laitteina.\\

  Tutkimus tehdään kirjallisuuskatsauksena. Kirjallisuuskatsaus valittiin tutkimussuunnitelmaksi siksi, sillä puettavista teknologioista on tehty paljon tutkimusta.
   Tutkimusstrategia on myös soveltuva siksi, että aihe on laaja, ja siinä on yhdisteltävä useita eri lähteitä. Yhden tutkimuksen tekeminen aiheesta on käytännössä mahdotonta,
    sillä aihe sisältää tutkimuksia useista eri mittaustavoista ja mittaustilanteista.
   
  
  

\chapter{Puettavat teknologiat}

\section{Puettavan teknologian määritelmä}
  Godfrey ym. (2018) määrittelevät artikkelissaan, että puettavat teknologiat tarkoittavat laitteita, jotka ovat suoraan suoraan tai irtonaisesti
   kiinni ihmisessä. \parencite{godfrey2018z}. Suoraan kiinni olevat laitteet ovat esimerkiksi uhreilukelloja,  Godfrey ym. (2018) antavat
    Irtonaiset puettavat teknologiat tarkoittavat yleensä puhelimia \parencite{godfrey2018z}. Yasar (2022) kertoo artikkelissaan, että
      Puettavia teknologioita voi olla monessa eri muodossa, ne voivat olla esimerkiksi koruja, lisävarusteita, lääketieteellisiä laitteita, vaatteita tai
      vaatteisiin liitettäviä laitteita muodossa \parencite{Yasar_what_wearable}.
    
  Puettavat teknologiat voidaan jakaa kahteen kategoriaan. Ensisijaisiin, eli laitteisiin jotka toimivat itsenäisesti
   ja yhdistävät muita laitteita toisiinsa \parencite{godfrey2018z}.  Tällaisia ovat esimerkiksi sykemittarit, puhelimet ja älysormukset.
    Toinen kategoria on toissijaiset, eli sellaiset laitteet jotka tarvitsevat toimiakseen ensisijasen laitteen, johon ne lähettävät dataa, 
     nämä mittaavat jotain tiettyä arvoa. Tällaisia ovat esimerkiksi rinnan ympärillä puettava sykevyö \parencite{godfrey2018z} \\

  \section{Äly- ja urheilukellot}
  Älykellot ovat suosituin laite puettavien teknologioiden kategoriassa \parencite{godfrey2018z}. Gregersen määrittelee Britannica
   tietosanakirjan artikkelissa älyjellojen tarkoittavan puhelimenkaltaisia laitteita, joita puetaan ranteessa \parencite{Gregersen_watch_2023}.
    Älykelloissa on myös yleensä joukko erilaisia antureita \parencite{rawassizadeh_wearables_2014}.\\

  The free dictionary määrittelee urheilukellon tarkoittavan tietokonepohjaista älykelloa, joka on kestävästi valmistettu, vedenkestävä ja siinä on ominaisuuksia kuten
    sykemittari sekä muita urheiluun, liikkumiseen ja tervetyteen liittyviä seuranta ominaisuuksia. \parencite{sportswatch_tfd}. Urheilukellojen ero älykelloon on siis se,
     että nämä ovat usein kestävämmin valmistettu, ja tarjoavat ensisijaisesti hyvinvoinnin ja urheilun seuraamiseen liittyviä ominaisuuksia. Erot ovat siis pienet, 
      ja esimerkiksi urheilukelloja valmistava Garmin käyttää sivuillaan kelloistaan "smart watch" termiä \parencite{garmin_site}. \\
  
\section{hyvinvointi}
Hyvinvoinnin määritelmä lyhyesti.
  % \section{älysormukset}
  % \section{sykevyöt}
  % \section*{mobiilisovellukset}


  \chapter{Älykellojen mittaukset}
  Tässä luvussa käydään läpi erilaisia mittauksia, mitä älykellot tekevät ja kuinka tarkkoja ne ovat. Lukuun ei ole kuitenkaa sisällytetty kaikkia älykellojen mittauksia, vaan
   ne joita kandidaatintutkielmani käsittelee.

  \section{Uni}
  Mitä mitataan (syke yms.) ja algoritmit jotka laskevat unen vaiheita, vertaillaan esim. lääketieteessä käytettyihin laitteisiin.
  Lähteitä esim. 
  Effect of Wearable Technology Combined With a Lifestyle Intervention on Long-term Weight Loss: The IDEA Randomized Clinical Trial \parencite{jakicic_effect_2016}
  Understanding continued smartwatch usage: the role of emotional as well as health and fitness factors \parencite{siepmann_understanding_2021}\\

  Riittävä unensaanti on tärkeä terveyden ja hyvinvoinnin ylläpitämiseksi. Watsonin sekä Andrewin artikkelissa (2017) todetaan unella olevan merkitystä niin fyysiseen kehitykseen,
   emootioiden hallintaan, kognitiiviseen suorituskykyyn ja muutenkin elämänlaatuun \parencite{watson_sleep_2017}. 
    Artikkelissa (2017) kerrotaan, että urheilijat arvioivat unensa laatua ja kestoa heikosti \parencite{watson_sleep_2017}. Unen mittaus ominaisuus on siis varsinkin urheilujoille
     tärkeä ominaisuus.
         
  \section{Paikanninjärjestelmä}
  Äly- ja urheilukellot käyttävät sijainnin, urheilusuorituksen pituuden ja nopeuden mittaamiseen Global navigation satellite system (GNSS) nimistä satelliittinavigointiverkostoa. 
   GNSS:ään lukeutuu Euroopan Galileo järjestelmä, Yhdysvaltojen Global Navigation System (GPS), Venäjän Glosnass sekä kiinan BeiDou satelliittipaikannusjärjestelmnät \parencite{hofmann2007gnss}.
    Joissain tutkimuksissa puhutaan GPS paikantimen tarkkuudesta, toisissa GLOSNASS:ista, ja toisissa taas viitataan yleisesti GNSS:ään. Puhuttaessa GNSS:stä tai GPS:stä tarkoitetaan kuitenkin
     samaa älykellon paikannin sirua, ainostaan salliitit joihin yhdistetään muuttuvat. \\

  
  GPS mittaamisen tarkkuuteen vaikuttaa lukuisia ulkopuolisia seikkoja. Cummins ym. (2013) artikkelissa kerrotaan, että laitteen päivitys frekvenss, vaikuttaa sijainnin mittaamisen tarkkuuteen.
   Mitä suurempi frekvenssi, sen tarkempi sensori on. \parencite{cummins_global_2013} Baranskin ja Strumillon artikkelit mainitsevat artikkelissaan (2012), että tarkkuuteen vaikuttaa myös
    se, onko GPS sirun lähellä korkeita rakennuksia ja onko taivas selkeä \parencite{baranski_enhancing_2012}. \\

  Gilgen-Ammann ym. toteavat artikkelissaan (2020), että niin ammatti kuin amatööri juoksijat nojautuvat pitkälti urheilukellon GNSS:n antamaan lukemaan juostun matkan pituudesta,
    ja urheilukellojen sijainnin :n tarkkuudesta on kuitenkin tehty vähän tieteellistä tutkimusta. GNSS mittaus on kuitenkin käytössä useassa lajissa, ja olisi tärkeä saada tietoa siitä,
     onko mittaukset luotettavia. Gilgen-Ammann tutkivat artikkelissaan (2020) Applen, Coroksen, Polarin ja Suunnon urheilukellojen GPS mittauksen tarkkuutta 400-4000 metrin juoksu,
      kävely sekä pyöräilysuorituksissa. Näistä kelloista Polar oli ainoa, jonka mittauksissa oli keskimäärin alle 5\% virhe mittauksissa.
       Keskimäärin urheilukellojen GPS mittauksissa on noin 3-6\% virhe. \parencite{gilgen-ammann_accuracy_2020} \\
    
  Johanssom ym. tutkivat artikkelissaan (2020) urheilukellojen GPS:n tarkkuutta ultramaraton juoksuksissa. Juoksuretin pituus oli 56 kilometriä ja reitti sisälsi 800 nousumetriä.
   Tutkimuksen mukaan GPS laitteiden tarkkuus oli tarkimmillaan $\SI{0.6 \pm 0.3}{\percent}$ ja epätarkimmillaan $ \SI{1.6 \pm 0.9}{\percent} $. (Mediaani $\pm$ IQR)(viittaus taulukkoon)
    Artikkelissa todettiin myös, että urheilukellot ovat pätevä tapa mitata juostua matkaa. \parencite{johansson_accuracy_2020}\\

  Johtopäätös?
  \section{Korkeudenmittaus}
    Noustut metrit treenin aikana on tärkeä osa urheilijan kokonais työmäärää, siksi älykellojen korkeuden mittaamisen tarkkuutta on hyvä tutkia \parencite{ammann_accuracy_2016}.
     Barometrien tarkkuudesta äly- tai urheilukelloissa on kuitenkin suppeasati tutkumusata. Äly- ja urheilukellot käyttävät korkeuden mittaamiseen barometristä sensoria tai GPS
      paikantimen antamaa informaatiota \parencite{ammann_accuracy_2016}. Barometrinen sensori arvioi korkeuden mittaamalla ilmanpainetta \parencite{aroganam2019review}.
       GPS taas antaa korkeustiedon paikannuksesta saatujen karttatietojen avulla. On myös mahdollista, että kello käyttää GPS ja barometrin kombinaatiota
        korkeuden mittaamiseen. Tällä tavalla kello yrittää korjata barometrin mahdolliset ulkoisista asioista, kuten sääolosuhteista johtuvat mittausvirheet. (viite kaiva jostai).\\
        
    KIRJOITA MIKSI BAROMETRI VOI OLLA EPÄTARKKA JOISSAIN TILANTEISSA EDELLISEEN KAPPALEESEEN + kirjoitanko barometrin tarkkuudesta sisällä?\\

    Ammann ym. tutkivat älykellojen korkeusmittauksen tarkkuutta kolmella erilaisella juoksureitillä. ensimmäinen reitti oli tasainen 400 metrin juoksurata.
     Toinen reitti oli 2490 metriä, jossa nousua oli 90 metriä. Kolmas reitti oli silmukka, jossa yhdellä kerralla oli nousua 30 metriä. Tutkimuksessa tätä reittiä kutsuttiin "mäkiseksi reitiksi"
      Osallistujat juoksivat jokaisen reitin kolmella eri nopeudella. Testaamiseen käytettiin Garmin ForerunnerXT, Polar RS800XC, sekä Suunto Ambit2 urheilukelloja. Näistä kelloista Garmin ja
       Suunto käyttivät korkeuden mittaamiseen barometrin sekä GPS yhdistelmää. \parencite[.]{ammann_accuracy_2016} Tutkimus toteutettiin kolmen kuukauden aikana, jotta sääolosuhteet voitiin ottaa huomioon.
         Kellojen kalibroitiin ennen jokaista testiä vastaamaan kyseistä korkeutta merenpinnan yläpuolella, jossa testaajat olivat \parencite{ammann_accuracy_2016}.\\

    Tutkimuksen mukaan kellot aliarvioivat mäkisellä reitillä nousumetrejä alimmillaan 3,3\% suurimmillaan 9,8\%. Tasaisella reitillä mittaukset olivat suhteellisen tarkkoja.
     Mittaukset erosivat todellisiin nousumetreihin tasaisella reitillä alimmillaan 0,0\% ja suurimmillaan 0,4\%, joten mittaukset olivat todella tarkkoja tasaisella reitillä \parencite{ammann_accuracy_2016}.
      Artikkelissa todettiin, että nousumetrien arviointien epätarkkuus saattoi johtua esimerkiksi käden heilumisesta juoksun aikana, sääolosuhteiden muutoksisesta tai GPS signaalin vaihtelevasta vahvuudesta \parencite{ammann_accuracy_2016}.
       On myös tärkeä mainita, että tutkimuksessa käytettävät kellot olivat julkaistu vuosina 2008, 2012 ja 2013. Tekniikka on voinut siis kehittyä tähän mennessä.\\


  \section{Syke} 

  Sykkeen mittaamiseen käytetään pääosin kahta erilaista teknologista menetelmää. Ensimmäinen metodi on elektrokardiografia (EKG, eng. ECG), jossa mitataan sydämmen lyönnin
   synnyttämiä sähköaaltoja \parencite{noauthor_heart_nodate}. Käytän tässä tutkielmassa jatkossa mittaustavasta lyhennettä EKG.
    Toinen metodi taas on fotopletysmografinen (Optinen, FPG, eng. PPG), jossa mitataan infrapunavaloa käyttäen valtimoiden supistumista ja laajenemista \parencite{noauthor_heart_nodate}.
     \cite{noauthor_heart_nodate} mukaan valon avulla voidaan myös arvioida veren happipitoisuuden määrää. Käytän tässä tutkielmassa jatkossa termiä optinen viitaten fotopletysmografiseen
      mittaukseen. EKG mittaamista käytetään yleensä rinnan ympärillä puettavissa sykevöissä, kun taas optista tekniikkaa ranteesessa puettavissa laitteissa \parencite{noauthor_heart_nodate}.
       Optisten sensoreiden tarkkuutta tutkiessa kriteerinä käytetään usein Polarin EKG mittausta käyttäviä sykevöitä, sillä ne ovat todettu olevan hyvin lähellä
        lääketieteessä käytettävien elektrodiagrammin mittauksia, varsinkin alhaisen, keskisuuren ja jopa suuren harjoitteluintensiteetin aikana \parencite{gilgen-ammann_rr_2019, nelson_accuracy_2019}.\\
  
  Sykkeen mittaamisen tarkkuuteen äly- ja urheilukelloissa löytyy eniten tutkimusta verrattuna muihin mitattaviin arvoihin.
   \cite{wang_accuracy_2017} tutkivat artikkelissaan miten Polar H7 sykevyö, sekä neljä erilaisen älykellon sykemittaukset vertautuivat lääketieteessä käytettävään
    elektrodiagrammin sykemittaukseen. Osallistujien sykettä mitattiin elektrodiagrammin avulla siten,
      että heillä oli neljä johtoa kiinnitettyinä elektrodeihin jokaisessa raajassa. Polar H7 sykevyön sekä kaksi satunnaista älykelloa molemmissa ranteissa.
       Osallistujat juoksivat matolla 2, 3, 4, 5 ja 6 mailin nopeuksilla, jokaisella nopeudella kolmen minuutin ajan \parencite{wang_accuracy_2017}.
        Tutkimuksen mukaan laitteet korreloivat elektrodiagrammin mittauksiin seuraavan kaavion mukaisesti.\\

  \begin{table}[h]
    \begin{center}
      \begin{tabular}{||c c c c c||} 
       \hline
       Polar H7 & Apple Watch & Mio Fuse & Fitbit Charge HR & Basis Peak \\ [0.5ex] 
       \hline\hline
       .99 (.987-.991) & .91 (.884-.929) & .91 (.882-.929)& .84 (.791-.872)& .83 (.779-.865)\\
       \hline
      \end{tabular}
    \caption{Älykellojen sykemittauksen tarkkuuden korrelaatio elektrodiagrammin sykemittaukseen \parencite{wang_accuracy_2017}}
      \end{center}
    \end{table}

  Laitteet olivat levossa tarkempia kuin harjoittelun aikana \parencite{wang_accuracy_2017}. Tämän näkee kuviosta vertaamalla korrelaation ylärajaa sen alarajaan.
   Polar H7 sykevyön mittasi siis sykettä tutkimuksen mukaan hyvin tarkasti, korreloiden .99 elektrodiagrammin mittauksiin \parencite{wang_accuracy_2017}.
    Kuten taulukosta huomataan, kahden kellon mittausten korrelaatio oli keskimäärin jopa alle .85, joten niiden sykemittaukset eivät olleet kovin tarkkoja.\\
  
  \cite{nelson_accuracy_2019} tutkivat artikkelissaan sitä, miten tarkasti Apple Watch 3, sekä Fitbit Charge 2 mittasivat sykettä jokapäiväisessä elämässä.
   Tutkimuksessa osallistuja pitivät kelloa 24 tunnin ajan ranteessaan eläen normaalia arkea, ja mittauksia verrattiin tutkimuksilla standardoidun
    vertailulaitteen tuloksiin \parencite{nelson_accuracy_2019}. Heidän tulokset olivat hieman erilaisia \cite{wang_accuracy_2017} artikkelin tuloksiin.
     Apple Watch 3 keskipoikkeama EKG mittarin tuloksiin oli 3,01\% juostessa, kävellessä 4,64\%, istuessa 7,21\%, nukkuessa 3,12\% ja keskimäärin 24 tunnin aikana ero oli 5.86\%.
      \parencite{nelson_accuracy_2019}. Fitbit kellon keskipoikkeama EKG mittariin oli juostessa 9,88\%, kävellessä 9.21\%, nukkuessa 3,36\%,
       istuessa 6,93\% ja keskimäärin 24 tunnin aika 5,96\% \parencite{nelson_accuracy_2019}.\\

  Tuloksissa erikoista on se, että Apple Watchin tarkkuus parani juostessa, verrattuna siihen mitä se oli istuessa. \cite{wang_accuracy_2017} tutkimuksessa
   mittausten tarkkuus heikkeni harjoittelun aikana.
    
       


  
   
     


  \section{Askeleet}
  Askelmittari ja sen tarkkuus
  
  \section{Tarkkuus}
  Tähän tiivistelmä siitä, miten tarkkoja tai epätarkkoja edellä mainitut mittarit ovat.
  
  \chapter{Älykellojen vaikutus hyvinvointiin}
  Mitä vaikutuksia älykelloilla on ihmisten hyvinvointikäyttäytymiseen. Auttaako laitteet lisäämään liikuntaa, enemmän unta tai terveellisempiä elämäntapoja ylipäätään?
  \section{Positiiviset}
  \section{Negatiiviset vaikutukset}
  
  \chapter{Johtopäätökset}

% The last chapter of a thesis is the Conclusion (some authors use
% Conculsions, instead).  Keep it short, and discuss what one can
% conclude about the thesis statement or research question given in the
% Introduction, in light of all that has been written in the thesis.
% The Conclusion is also the place to discuss any limitations and
% weaknesses of the thesis (especially those that cast doubt on the
% reliabliity of the results given in the thesis), if they have not been
% already discussed, for example in a Discussion chapter.  It is also
% customary to state, what further research might be beneficial in light
% of this thesis.

% If the Conclusion threatens to become too long, it is a good idea to
% split the interpretation of the results into its own chapter, often
% called Discussion, making Conclusion short and sweet.

% After Conclusion, there is the bibliography, indicated by the
% \string\printbibliography\

\printbibliography



\end{document}
