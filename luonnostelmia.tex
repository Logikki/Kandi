\documentclass[12pt]{article}
\usepackage[utf8]{inputenc}
\usepackage[finnish]{babel} % Kieli: Suomi
\usepackage{mathptmx}
\usepackage{setspace}
\usepackage{microtype}
\usepackage{apacite}
\usepackage{parskip}
% \usepackage{cite}
\usepackage{natbib}
\usepackage[margin=1.25in]{geometry}
\date{19.01.2022}
\author{Roni Koskinen}
\begin{spacing}{1.5}
    \begin{document}

    \section{esittely}
    Puettavat teknologiat voidaan jakaa kahteen kategoriaan. Ensisijaisiin, eli laitteisiin jotka toimivat itsenäisesti
     ja ovat yhdistävät muita laitteita toisiinsa. Tällaisia ovat esimerkiksi sykemittarit tai puhelimet. Toinen kategoria
      on toissijaiset, eli sellaiset laitteet jotka tarvitsevat toimiakseen ensisijasen laitteen, johon ne lähettävät dataa, 
      nämä mittaavat jotain tiettyä arvoa. Tällaisia ovat esimerkiksi rinnan ympärillä puettava sykevyö \cite{godfrey2018z} \\

    Älykellot ovat suosituin laite puettavien teknologioiden kategoriassa \cite{siepmann_understanding_2021}.

    \section{hyödyt}
    Tutkimuksessa \cite{rapp_self-tracking_2020} todettiin, että puettavan teknologian avulla pystyi tekemään nopeita toistaja
     missä tahansa sen sijaan, että täytyisi mennä radalle harjoittelemaan.\\
    \section{Motivaatio}
    Urheilukellojen amatööriurheiluja käyttäjät kertoivat, että sykemittarilla itsensä seuraaminen ja urheiludatan kerääminen motivivoi urheilemaan ja parantamaan suoritusta.
     Urheilukello toimii ikäänkuin valmentajana, joka seuraa suoritusta. \cite{rapp_self-tracking_2020} Laite voi kuitenkin muuttaa urheilusuorituksen
      luonnetta stressaavammaksi, sillä urheiluaktiviteetti mielletään työksi sen sijaan, että se tehtäisiin sisäisten motivaatiotekijöiden kautta.
       \cite{rapp_self-tracking_2020}\\

    \section{Hankkiminen}
    Sykemittarin hankkimiseen vaikuttaa moni tekijä. On kuitenkin löydetty asioita, jotka johtavat sykemittarin hankintaan. Esimerkiksi Ne juoksijat, jotka osallistuvat 
     enemmän kuin yhteen juoksutapahtumaan vuodessa käyttävät suuremmalla todennäköisyydellä urheilukelloja \cite{janssen2017uses}. Urheilukellon ostamiseen vaikuttaa
      myös sosiaaliset vaikutteet. Esimerkiksi ystävän hankkima urheilukello saattaa motivoida hankkimaan sellaisen itselleen \cite{grinblatt_social_2008}.
       Moni hankkii myös itsensä seuranta laitteen pelkästään uteliaisuudesta \cite{baumgart2016role}.

    Sykemittarin hankkiminen uteliaisuudesta on yhteydessä suurempaan rahankäyttöön hyvinvointimittaamis palveluihin ja laitteisiin \cite{baumgart2016role}.\\

    
    \section{uni}
    Watsonin ja Andrewin artikkelissa (2017) kerrotaan, että urheilijat arvioivat unensa laatua ja kestoa heikosti \parencite{watson_sleep_2017}

    \bibliographystyle{apacite}
\bibliography{references}
\end{spacing}
\end{document}
